% Copyright 2004 by Till Tantau <tantau@users.sourceforge.net>.
%
% In principle, this file can be redistributed and/or modified under
% the terms of the GNU Public License, version 2.
%
% However, this file is supposed to be a template to be modified
% for your own needs. For this reason, if you use this file as a
% template and not specifically distribute it as part of a another
% package/program, I grant the extra permission to freely copy and
% modify this file as you see fit and even to delete this copyright
% notice. 

\documentclass{beamer}
\usepackage{tikz}
\usepackage{float}
\usetikzlibrary{topaths,calc}
\usetheme{Frankfurt}
\usecolortheme{rose}

\addtobeamertemplate{frametitle}{
   \let\insertframetitle\insertsectionhead}{}
\addtobeamertemplate{frametitle}{
   \let\insertframesubtitle\insertsubsectionhead}{}


\makeatletter
  \CheckCommand*\beamer@checkframetitle{\@ifnextchar\bgroup\beamer@inlineframetitle{}}
  \renewcommand*\beamer@checkframetitle{\global\let\beamer@frametitle\relax\@ifnextchar\bgroup\beamer@inlineframetitle{}}
\makeatother

\makeatletter
\newenvironment<>{proofs}[1][\proofname]{%
    \par
    \def\insertproofname{#1\@addpunct{.}}%
    \usebeamertemplate{proof begin}#2}
  {\usebeamertemplate{proof end}}
\makeatother

\title{Matching Problem Generalizations on Hypergraphs}
\subtitle{or ``Why You Hate Your Roommate''}

\author{Dominick Twitty (\texttt{dkt36})\\
\and Kelvin Luu (\texttt{kl583})\\
\and Jeff Tian (\texttt{yt336})
}

\date{Networks II Final Project Presentation}

\begin{document}

\begin{frame}
  \titlepage
\end{frame}

\section{Problem Statement}
\begin{frame}
\begin{itemize}
\item Networks II studies matchings on bipartite graphs - two agents from different sets get matched

\item We can generalize matchings to deal with an arbitrary number of agents and sets

\item Using hypergraphs, we can encode problems where agents have preferences over sets of items
\end{itemize}

\end{frame}

\subsection{A Driving Example}
\begin{frame}
\begin{itemize}
\item At Cornell, freshmen fill out a questionnaire to determine roommate compatibility, and then they are placed into rooms to try to maximize compatibility.

\item Some people love their roommate, but there are lots of people who'd rather switch with someone else.

\item How can we provide students with better matches?
\end{itemize}
\end{frame}


\subsection{Detailed Formulation}
\begin{frame}
We define the following problem:
\begin{block}{The Binary Freshman Roommates Problem}
\begin{description}
\item[Inputs:]
\begin{itemize}
\item $S$ : A set of students who need rooms
\item $C_s$: The set of roommates $s$ can live with
\item $L$ : A list of available room sizes.
\end{itemize}

\item[Objective:]
Output a room assignment $M$ such that the maximum number of students have compatible roommates.
\end{description}
\end{block}
\end{frame}


\begin{frame}
We also define the problem given rank-order preferences
\begin{block}{The Rank-Order Freshman Roommates Problem}
\begin{description}
\item[Inputs:]
\begin{itemize}
\item $S$ : A set of students who need rooms
\item $P(s)$ : Preferences for each student $s$ over all subsets containing $s$.
\item $L$ : A list of available room sizes.
\end{itemize}

\item[Objective:]
Output a stable or Pareto-efficient room assignment.
\end{description}
\end{block}
\end{frame}

\section{Analysis}

\subsection{Intractability}
\begin{frame}
In general, this is a very difficult problem
\begin{block}{Claim}
The Binary Freshman Roommates Problem is NP-Complete
\end{block}
\begin{proof}
We reduce from the clique problem. Given a graph $G = (V, E)$ and a number $k$, we encode the following FRP:
\begin{itemize}
\item $S = V$
\item $C(v) = \{u : (u, v) \in E\}$
\item $L$ is one room of size $k$, and $|V| - k$ roomss of size 1.
\item Return "Yes" if all students are assigned a room, else "No"
\end{itemize}
\end{proof}
\end{frame}

%\begin{frame}{NP-Completeness}
%\begin{proofs}[\proofname\ (Cont.)]
%\begin{description}
%\item[Yes to Yes] If there is a clique of size $k$, then all vertices in that clique will list each other as compatible. These clique vertices fill the size $k$ room, and all other vertices fill the single rooms.
%
%\item[No to No] If the largest clique is of size less than $k$, then the big room can't fill up. However, there aren't enough singles to take cover all the other students, so some students must be left out. 
%\end{description}
%\end{proofs}
%\end{frame}

\begin{frame}{N}
\begin{itemize}
\item Unless P = NP, any algorithm for solving the binary problem will not be polynomial. 
\item This is a very hard problem, and even if you could find a maximum allocation, you are still obligated to assign incompatible students to whatever slots are left.
\item Can we still reason about this problem given rank-order preferences?
\end{itemize}
\end{frame}

\subsection{Good Matches}

\begin{frame}
What is stability in this situation?
\begin{itemize}
\item Every student has a preference over sets of roommates.
\item A matching $M$ is blocked by $T \notin M$ if all students in $T$ prefer $T$ to their current set.
\item This is a generalized definition of a blocking pair.
\end{itemize}

Stability is not always guaranteed. For instance, the stable roommates problem is an instance of FRP with $2n$ students and $n$ rooms of size 2. It is known that there are instances of SRP with no solution, therefore not every FRP has a stable solution.
\end{frame}

\begin{frame}
How about Pareto-efficiency?
\begin{itemize}
\item Definition of Pareto-efficiency generalizes trivially.
\item It can be shown using the pigeonhole principle that every FRP has a Pareto-efficient matching.
\item Pareto-efficiency is a nice way to promise a student she can't switch rooms without strictly hurting someone else.
\end{itemize}
\end{frame}

\section{Algorithms}
\subsection{Solving the Binary Problem}
\begin{frame}
\begin{itemize}
\item Given $l$ room slots, a maximum room size $k$ and $n$ students, we have constructed an algorithm to determine the maximum number of students that can be allocated compatibly.
\item The algorithm is based on max flow and has runtime $\mathcal{O}(n^{3k})$
\item A brute-force approach takes something like $\displaystyle \sum_{r = 1}^n r! \binom{n}{r} \binom{l}{r}$
\item The addition of the list of rooms keeps us from reducing to Set Packing, and also complicates formulations with Integer Linear Programming.
\end{itemize}
\end{frame}

\subsection{Generating Pareto-Efficient Matchings}
\begin{frame}
\begin{itemize}
\item Clearly, this problem is too hard to be interesting even by NP-Hard optimization standards.
\item The rank-order version of the problem lets us make assumptions that simplify the problem.
\item For example, we can safely require that the number of room slots equal the number of students while still maintaining individual rationality.
\end{itemize}
\end{frame}

\begin{frame}
\begin{itemize}
\item In order to Pareto-improve a matching, one student must switch rooms, which implies kicking someone else out of their room.
\item This trade can happen along a cycle, so long as the number of students in each room remains constant.
\item Generally, we know we can improve a matching if some coalition of students owns a set of rooms, and there is a way to reassign them to those rooms such that everyone benefits.
\end{itemize}
\end{frame}

\begin{frame}
We propose the following algorithm, which attempts to Pareto-dominate an existing matching $M$
\begin{block}{The FRP HyperCycle Algorithm}
\begin{itemize}
\item Create a hypergraph $G$. Add a vertex for every student and every room.
\item For every existing group $T \in M$ with room $r$, add a \emph{``dark''} hyperedge surrounding $T$ and $r$.
\item For every blocking set $T' \notin M$, create a \emph{``light''} hyperedge surrounding $T'$ and $r$ for every compatibly-sized room $r$.
\end{itemize}
\end{block}
\end{frame}

\begin{frame}
\begin{block}{The FRP HyperCycle Algorithm (Cont.)}
\begin{itemize}
\item Ask if there is some set $D$ of dark edges and a set $L$ of light edges such that:
\begin{itemize}
\item All dark edges are pairwise disjoint
\item All light edges are pairwise disjoint
\item $D$ and $L$ cover the same vertices
\end{itemize}
\item If no such $D$ and $L$ exist, then $M$ is Pareto-efficient.
\item If they do exist, for each $(T', r) \in L$, assign group $T'$ to $r$ to Pareto-improve $M$. Rinse and repeat.
\end{itemize}
\end{block}
\end{frame}

\begin{frame}
\begin{itemize}
\item The FRP HyperCycle Algorithm takes inspiration from the TTC algorithm and the Pareto-efficient roommates algorithm.
\item Every dark edge represents a current allocation. Every light edge represents an opportunity to improve.
\item If $D$ and $L$ exist, then $D$ provides students and rooms as inputs, and $L$ provides a better way to reallocate them.
\item If they don't exist, then there is no subset of students and rooms that can be reordered, thus the matching is Pareto-efficient.
\end{itemize}
\end{frame}

\section{Conclusion}
\subsection{Limitations}
\begin{frame}
We are well aware of the limitations of our results.
\begin{itemize}
\item As was fully expected, our generalized algorithms will have monstrous complexities.
\item For example, we solve the maximum bipartite matching problem in $\mathcal{O}((n + m)^3)$. 
\item The rank order version of this problem is also very intense, requiring long preference lists.
\item We hope to include ``useful'' solutions as well as optimal ones.
\end{itemize}

\end{frame}

\subsection{Open Questions}
\begin{frame}
There are still questions we will hopefully have time to explore in our final report
\begin{itemize}
\item Can we extend the HyperCycle algorithm to include more housing slots than students?
\item Can we deal with multiple sets, like how males and females cannot live together?
\item Is there some heuristic for generating a ``pretty good'' allocation? For example, what if you only allow people to swap rather than trade on a cycle?
\end{itemize}
\end{frame}

\subsection{Signing Off}
\begin{frame}
\begin{center}
\Huge Thank You
\end{center}

\end{frame}




\end{document}


