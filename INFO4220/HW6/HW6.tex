\documentclass[12pt]{article}
\usepackage{fullpage}
\usepackage{amsmath, amsthm, amssymb}
\usepackage{enumerate}
\usepackage{mathtools}
\usepackage{nicefrac}

\newtheorem*{claim}{Claim}
\newtheorem{lemma}{Lemma}
\newtheorem{theorem}{Theorem}
\newtheorem*{definition}{Definition}
\begin{document}
\title{INFO 4220 Homework 6}
\author{Dominick Twitty}
\date{}
\maketitle

\section{Bayesian Inference on eBay}
Let $G$ be the event that a seller is good, $B$ be the event that the seller is bad, and $N$ be the event that a buyer has a bad experience with a seller. We are given the following information
\begin{align*}
P(G) = P(B) = 0.5 && P(N | B) = 0.3 && P(N | G) = 0.01 
\end{align*}

\begin{enumerate}[(a)]
\item Using the formula for Bayesian inference given multiple observations, we see
\begin{align*}
P(B | \bar N^{10}) &= \frac{P(\bar N | B) ^ {10} P(B)}{P(\bar N | B) ^ {10} P(B) + P(\bar N | G) ^ {10} P(G)}\\
&= \frac{(0.7)^{10} (0.5)}{(0.7)^{10} (0.5) + (0.99)^{10} (0.5)}\\
&= 0.0303
\end{align*}
\item
\begin{align*}
P(B | \bar N^{1000} + N^{10}) &= \frac{P(\bar N | B) ^ {1000} P(N | B)^{10}P(B)}{P(\bar N | B) ^ {1000} P(N | B)^{10}P(B) + P(\bar N | G) ^ {1000} P(N | G)^{10}P(G)}\\
 &= \frac{(0.7)^{1000}(0.3)^{10} (0.5)}{(0.7)^{1000}(0.3)^{10} (0.5) + (0.99)^{1000} (0.01)^{10} (0.5)}\\
&= 1.71 \times 10^{-136}
\end{align*}
\begin{align*}
P(B | \bar N^{1000} + N^{50}) &= \frac{P(\bar N | B) ^ {1000} P(N | B)^{50}P(B)}{P(\bar N | B) ^ {1000} P(N | B)^{50}P(B) + P(\bar N | G) ^ {1000} P(N | G)^{50}P(G)}\\
 &= \frac{(0.7)^{1000}(0.3)^{50} (0.5)}{(0.7)^{1000}(0.3)^{50} (0.5) + (0.99)^{1000} (0.01)^{50} (0.5)}\\
&= 2.08 \times 10^{-77}
\end{align*}

\item 
\begin{align*}
&P_A(\bar N) = P(\bar N | G) P_A(G) + P(\bar N | B) P_A(B) = (0.99)(0.9697) + (0.7)(0.0303) = 0.981\\
&P_B(\bar N) = P(\bar N | G) P_B(G) + P(\bar N | B) P_B(B) = (0.99)(1) + (0.7)(0) = 0.99
\end{align*}
The buyer will be willing to pay $0.009v$ more to buy from $B$ than $A$.

\end{enumerate}

 

\section{Equilibrium Fractions}
\begin{enumerate}[(a)]
\item
\begin{enumerate}[i.]
\item Let $h$ be the fraction of good cars sold. Given that $g = 0.8$, the value buyers are willing to pay for any car is $13.2$, which is higher than $v_g$. Thus all sellers are willing to put their car up for sale and all good cars are sold ($h = g$). When $g = 0.2$, the value buyers are willing to pay is $7.8$, which is below $v_g$. Thus owners of good cars are not willing to sell, and no good cars will be sold ($h = 0$). In both cases, 0 good cars sold is also an equilibrium.

\item In order for all cars to be sold, the price that buyers are willing to pay must be at least as great as a seller's value for a good car. This tells us
\[15 g + 6(1 - g) \geq 10\]
So when $g^* = \nicefrac{4}{9}$, for any $g \geq g^*$, there is an equilibrium where all good cars are sold ($h = g$).

\item We need the buyer's price to be at least 10. Then
\[ 10 \leq 0.6 w_g + 0.4(6)\]
Then $w_g$ needs to be at least $12.67$.
\end{enumerate}

\item Let $v_h$ be the seller's value for the highest-valued car. Then the expected seller value of a car is $\nicefrac{v_h}{2}$. In order for all cars to be sold, then the expected buyer value must exceed $v_h$. Then
\[\alpha \nicefrac{v_h}{2} \geq v_h\]
and $\alpha$ must be at least $2$ for all cars to be sold ($v_h = 1)$.  
\end{enumerate}


\section{Equilibria in Infinite Prisoners Dilemma}
\begin{enumerate}[(a)]
\item Given that both players always play $C$, then the total payoff for $P_2$ is
\[\sum_{k = 0}^{\infty} r \delta^k = \frac{r}{1 - \delta} = \frac{2}{1 - \delta}\]
\item As $P_2$ deviates at $N$ and $P_1$ is playing $G$, then $P_1$ will defect forever starting at $N + 1$. AS $P_2$ can no longer achieve $r$ or $t$, the best payoff is for $P_2$ to defect forever as well. As all outcomes are scaled equally, this choice is not dependent on the value of $\delta$.
\item The total payoff for $P_2$ is 
\[\frac{p \delta^{N + 1} -r (\delta^N - 1) + (1-\delta)\delta^N t}{1 - \delta} = \frac{\delta^{N + 1} -2 (\delta^N - 1) + 3(1-\delta)\delta^N}{1 - \delta}\]
\item For both players playing $G$ to be a Nash equilibrium, then we need $\delta$ such that there does not exist some time $N$ for which the payoff of deviating is higher than the payoff of cooperating.
\[\frac{r}{1 - \delta} \geq \frac{p \delta^{N + 1} -r (\delta^N - 1) + (1-\delta)\delta^N t}{1 - \delta}\]
\[\vdots\]
\[\delta \geq \frac{r - t}{p - t}\]
\[\delta \geq \frac{1}{2}\]
We showed in part (b) that assuming perfect play, once a player deviates they must continue to deviate given that the other is playing grim-trigger. However, if $\delta$ is small enough, then at some point a player could gain more from one step of deviation than they lose when the other player also deviates.
\end{enumerate}

\section{Specific Equilibria in Infinite Prisoners Dilemma}
We calculated above that $G$ is an equilibrium set of strategies if 
\[\delta \geq \frac{r - t}{p - t}\]
\begin{enumerate}[(a)]
\item $\delta \geq \frac{1}{4}$
\item $\delta \geq \frac{3}{4}$
\item $\delta \geq \frac{2}{5}$
\item Changing $s$ does not actually affect the equilibrium. Say that $P_2$ deviates first. Then the total payoff for $P_2$ must be greater than that of $P_1$, as $P_2$ gets one turn of $t$ payoff and $P_1$ gets one turn of $s < t$ payoff. Therefore, it is only the first deviant we need to stop from defecting. $\delta$ must still be greater than or equal to one half.
\end{enumerate}

\begin{align*}
 P(\o{G}|R^{10}) &= \frac{P(R|\o{G})^{10}P(\o{G})}{P(R)^{10}} \\
 &= \frac{(1-0.3)^{10}(0.5)}{((0.3)(0.5)+(0.01)(0.5))^10} \\
 &= \\
\end{align*}

\end{document}
