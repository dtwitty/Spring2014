\documentclass[12pt]{article}
\usepackage{fullpage}
\usepackage{amsmath, amsthm, amssymb}
\usepackage{enumerate}
\usepackage{mathtools}
\usepackage{multicol}
\usepackage{nicefrac}
\usepackage{graphicx}
\usepackage{float}
\newtheorem{lemma}{Lemma}
\newtheorem{theorem}{Theorem}
\newtheorem{claim}{claim}
\newtheorem*{lemma*}{Lemma}
\newtheorem*{theorem*}{Theorem}
\newtheorem*{claim*}{Claim}
\usepackage{subcaption}

\newcommand*{\Real}{\mathbb{R}}
\newcommand*{\Z}{\mathbb{Z}}
\newcommand*{\ord}{\mathrm{ord}}
\newcommand*{\inv}{^{-1}}


\begin{document}

\title{MATH 4320 Homework 3}
\author{Dominick Twitty}
\date{}
\maketitle

\section{Abelian Quotient Groups}
\begin{claim*}
Let $H$ be a subgroup of $G$. For all $a$ in $G$, $aH = H \iff a \in H$.
\end{claim*}
\begin{proof}
We prove both sides.
\begin{description}
\item[$\implies$] As $e$ is in $aH$, we have that $a$ is in $aH$. Because $aH = H$, we have that $a$ is in $H$.
\item[$\impliedby$] First, we show that $aH \subseteq H$. As $H$ is closed, we have that for all $h$ in $H$, $ah$ is also in $H$. Therefore, all elements of $aH$ are also in $H$. Next, we show that $H \subseteq aH$. For every $h$ in $H$, by definition we have that $ah$ is in $aH$. As we showed, $a$ is also in $aH$. As $aH$ is a group, $a\inv$ must also be in $aH$. So, $a\inv (ah) = h$ must be in $aH$. Therefore, every element of $H$ is also in $aH$.
\end{description}
\end{proof}

Let $C$ be the commutator subgroup of $G$.
\begin{claim*}
$G / C$ is abelian.
\end{claim*}
\begin{proof}
\begin{align*}
[x,y] C &= C\\
xy x\inv y\inv C &= C\\
x\inv y\inv C &= y\inv x\inv C\\
x\inv C \circ y\inv C &= y\inv C \circ x\inv C
\end{align*}
\end{proof}

\begin{claim*}
$G / M$ is abelian if and only if $C \subseteq M$.
\end{claim*}
\begin{proof}
Let $aM$ and $bM$ be members of $G/M$. We have
\begin{align*}
aMbM &= bMaM\\
abM &= baM\\
a\inv b\inv a b M &= M
\end{align*}
Thus, we have that $[a\inv, b\inv]$ is in $M$. Thus, $C \subseteq M$. Now suppose that for all $a, b$ in $G$, $[a,b]$ is in $M$.
\begin{align*}
ab a\inv b\inv M &= M\\
a\inv b\inv M &= b\inv a\inv M\\
\end{align*}
\end{proof}

\section{Subgroups of Finitely Generated Groups}
\begin{claim*}
Every finitely generated abelian group is the product of cyclic groups.
\end{claim*}
\begin{proof}
Suppose we have such a group $G = <a_1, \ldots, a_n>$. The generating set $A$ does not have to be the minimal generating set, just a finite set. We induct on the number of generators.
\begin{description}
\item[Base Case:] $<a_1>$ is a by definition a cyclic group, and is therefore a product of cyclic groups.
\item[Inductive Hypothesis:] Any abelian group generated by a finite set of size $k < n$ can be decomposed into a product of cyclic groups.
\item[Inductive Step:] We have
\begin{align*}
<a_1, \ldots, a_n> &= \{a_1^{k_1} \ldots a_{n - 1}^{k_{n - 1}} a_n^{k_n} | k \in \mathbb{Z}\}\\
                    &= \{a_1^{k_1} \ldots a_{n - 1}^{k_{n - 1}} | k \in \mathbb{Z}\}<a_n>\\
                    &= <a_1, \ldots, a_{n - 1}><a_n>
\end{align*}
\end{description}
As $<a_1, \ldots, a_{n - 1}>$ is generated by a finite set of size $n - 1 < n$, we can decompose $G$ into a product of cyclic groups.
\end{proof}
\begin{claim*}
Given a finitely generated abelian group $G$, $H \leq G$ is finitely generated.
\end{claim*}
\begin{proof}
A suitable proof could not be generated within the given time bounds.
\end{proof}


\section{Abelian Implications of a Cyclic Quotient Group}
\begin{claim*}
If $G / Z(G)$ is cyclic, then $G$ is abelian.
\end{claim*}
\begin{proof}
If $G/Z(G)$ is cyclic, then let $\tau$ be the single generating element of $G / Z(G)$. We have that as $\tau$ is a coset of $Z(G)$, there must exist some $t$ in $G$ with $tG = \tau$. So, every element of $G / Z(G)$ is of the form $\tau^i = (t^i)Z(G)$. 

Let $x$ and $y$ be members of $G$. Suppose we have $x \in (t^i)Z(G)$ and $y \in (t^j)Z(G)$. Then $x = t^i z_1$ and $y = t^j z_2$ for some $z_1, z_2$ in $Z(G)$. Remember that $z_1$ and $z_2$ commute with all members of $G$.
\begin{align*}
xy &= yx\\
t^i z_1 t^j z_2 &= t^j z_2 t^i z_1\\
t^i t^j z_1 z_2 &= t^j t^i z_2 z_1\\
t^{(i + j)} z_1 z_2 &= t^{(i + j)} z_1 z_2
\end{align*}
Finally, we must show that every $g$ in $G$ is a member of some $(t^i)Z(G)$. We have that $gZ(G) \in G/Z(G) = \tau ^ i$. Therefore, $g$ is in $gZ(G) = (t^i)Z(G)$. So, there must exist some $z \in (t^i)Z(G)$ with $g = t^i z$. Therefore, the above statements hold for all $x$ and $y$ in $G$. As we can show that $xy = yx$ for all $x$ and $y$ in $G$, we have that $G$ is abelian.
\end{proof}

\section{Disjoint Normal Subgroups}
\begin{claim*}
Let $H$ and $K$ be normal subgroups of $G$, with $H \cap K = \{1\}$. For all $x$ in $H$ and $y$ in $K$, $xy = yx$.
\end{claim*}
\begin{proof}
First, we have that as $H$ is normal, $y x y\inv \in H$. Following from this, we multiply by $x\inv$ on the right to show that $(y x y\inv) x\inv = [y, x] \in H$. Similarly, we can show that $ y (x y\inv x\inv) = [y, x] \in K$. So, for all $x$ in $H$ and $y$ in $K$, $[y, x] \in H \cap K$. As $H \cap K = \{1\}$, then $[y,x] = 1$. Therefore, $x$ and $y$ commute in $G$. 
\end{proof}

\section{Normal Subgroup Cross Quotient Isomorphism}
\begin{claim*}
Let $H$ be a normal subgroup of $G$. Let $K$ be a subgroup of $G$. $HK = KH$.
\end{claim*}
\begin{proof}
As $H$ is normal, we have that for all $k$ in $K$, $kH = Hk$. One can define $HK$ as 
\[ HK = \bigcup_{k \in K} Hk \]
One can define $KH$ as 
\[ KH = \bigcup_{k \in K} kH \]
As $kH = Hk$ for all $k$ in $K$, we have that $HK = KH$. Equivalently, for all $h$ in $H$ and $k$ in $K$, there exists some $h'$ and $k'$ such that $hk = k' h'$.
\end{proof}
\begin{claim*}
Let $M$ and $N$ be normal subgroups of $G$ with $G = MN$. $G / (M \cap N) \cong (G/M) \times (G/N)$. 
\end{claim*}
\begin{proof}
Let $\phi : G \rightarrow G/M \times G/N$ be a mapping with $\phi(g) = (gM, gN)$. First, we have that $\phi$ is a homomorphism (which can easily be shown using properties of cosets). If $g$ is in the kernel of $\phi$, we have that $(gM,gN) = (M, N)$. By the claim in problem 1, this implies that $g$ is in both $M$ and $N$, thus $g \in M \cap N$. This holds for all $g$ in $M \cap N$, so $\ker \phi = M \cap N$.

We argue that $\phi$ is surjective. Using the claim above, for any $g_1, g_2 \in G$, we can construct some $g$ such that $\phi(g) = (g_1M, g_2N)$. We have that $(g_1M, g_2N) = (n_1m_1M, n_2m_2N) = (n_1M, m_2N)$. If we let $g = m_2 n_1$, we have that $\phi(g) = (m_2 n_1 M, m_2 n_1 N) = (n_1M, m_2N) = (g_1M, g_2N)$. Therefore, $\phi$ is surjective.

By the First Isomorphism Theorem, $G / \ker \phi = G / (M \cap N) \cong (G/M) \times (G/N)$
\end{proof}

\section{Groups of Prime Square Order}
\begin{claim*}
Let $p$ be a prime and $G$ be a group of order $p^2$. Either $G \cong \Z_p \times \Z_p$ or $G$ is cyclic.
\end{claim*}
\begin{proof}
A suitable proof could not be generated within the given time bounds.
\end{proof}

\section{Doubly Transitive Permutation Action}
\begin{claim*}
For $n \geq 2$, $S_n$ acts doubly-transitively on $X = \{1,\ldots,n\}$. 
\end{claim*}
\begin{proof}
Clearly, $S_n$ acts transitively on $X$, as for all $x,y \in X$ we have that the transposition $\pi = (x\ y)$ satisfies $\pi(x) = y$. When we remove $x$ for $X$, we see that when $n \geq 2$, there will always exist some $z \in X - \{x\}$. As the permutation $(z\ y)$ is in $S_n$ for all $z,y \in X - \{x\}$, we have that $S_n$ acts doubly-transitively on $X$. In fact, this proof can be extended to show that  $S_n$ acts $n$-transitively on $X$.
\end{proof}

\section{Harry Potter and the Cartesian Products}
We list the conjugacy classes of
\begin{description}
\item[$\Z_2 \times S_3$:] We see a structure not unlike the cartesian product of the conjugacy classes \hfill
\begin{multicols}{2}
\begin{itemize}
\item $\{(0, 1)\}$
\item $\{(1, 1)\}$
\item $\{(0, (1\ 2)),(0, (1\ 3)),(0, (2\ 3))\}$
\item $\{(1, (1\ 2)),(1, (1\ 3)),(1, (2\ 3))\}$
\item $\{(0, (1\ 2\ 3)),(0,(1\ 3\ 2))\}$
\item $\{(1, (1\ 2\ 3)),(1,(1\ 3\ 2))\}$
\end{itemize}
\end{multicols}
\item[$\Z_3 \times A_4$:] We notice a similar pattern\hfill
\begin{itemize}
\item $\{ (0, 1) \}$, $\{ (1, 1) \}$, $\{ (2, 1) \}$
\item $\{ (0, (1\ 2)(3\ 4)), (1, (1\ 3)(2\ 4)), (1, (1\ 4)(2\ 3)) \}$
\item $\{ (1, (1\ 2)(3\ 4)), (1, (1\ 3)(2\ 4)), (1, (1\ 4)(2\ 3)) \}$
\item $\{ (2, (1\ 2)(3\ 4)), (2, (1\ 3)(2\ 4)), (2, (1\ 4)(2\ 3)) \}$
\item $\{ (0, (1\ 2\ 3)), (0, (1\ 4\ 2)), (0, (1\ 3\ 4)), (0, (2\ 4\ 3)) \}$
\item $\{ (0, (1\ 3\ 2)), (0, (1\ 4\ 3)), (0, (1\ 2\ 4)), (0, (2\ 3\ 4)) \}$
\item $\{ (1, (1\ 2\ 3)), (1, (1\ 4\ 2)), (1, (1\ 3\ 4)), (1, (2\ 4\ 3)) \}$
\item $\{ (1, (1\ 3\ 2)), (1, (1\ 4\ 3)), (1, (1\ 2\ 4)), (1, (2\ 3\ 4)) \}$
\item $\{ (2, (1\ 2\ 3)), (2, (1\ 4\ 2)), (2, (1\ 3\ 4)), (2, (2\ 4\ 3)) \}$
\item $\{ (2, (1\ 3\ 2)), (2, (1\ 4\ 3)), (2, (1\ 2\ 4)), (2, (2\ 3\ 4)) \}$
\end{itemize}
\end{description}

\section{Harry Potter and the Tri-Colored Necklaces}
One can generate 130 necklaces (according to the definition as unique up to rotation) of 6 beads and 3 distinct colors. Consider the group $\rho$ of rotations on a string of length 6, with 3 possible colors per position. We will call the set of these strings $X$. As all strings are invariant under 0-rotation, we have that $|X^1| = |X| = 3^6$. Next, all strings invariant under 1 or 5 rotations have the form $x_i = x_i - 1$, and are thus monochrome. So, $|X^{\rho ^ 1}| = |X^{\rho ^ 5}| = 3$. A similar line of thinking applies for strings invariant under 2 and 4 rotations, as this is the product of sub-strings of size 3 invariant under 1 rotation. So, $|X^{\rho ^ 2}| = |X^{\rho ^ 4}| = 3^2$. Finally, strings invariant under 3 rotations are really the concatenation of 2 identical strings of size 3. So, $|X^{\rho ^ 3}| = 3^3$. Using Burnside's lemma, we have that the number of orbits of $\rho$ on $X$, which is the same as the number of strings unique up to rotation, which is the number of necklaces, is \[\frac{1}{6}\sum_{i = 0}^5 |X^{\rho ^ i}| = 130\]

One notices that the group we use is a cyclic group. The problem states that we should be using a dihedral group. In this case, according to online sources, we are dealing with bracelets instead of necklaces. Similar thoughts on orbits apply. There are 92 such bracelets.

\section{Harry Potter and the Cube of Many Faces}
We consider the possible cube rotations and the number of cube states left invariant by each.
\begin{description}
\item[Identity:] All $r^6$ possible states remain unchanged.
\item[90 Degree Rotation About a Face:] The rotated faces always remain invariant, and the outer faces must all be the same color, which leaves $r^3$ unchanged states. There are 6 of these rotations.
\item[180 Degree Rotation About a Face:] As above, except there can now be 2 colors to choose from on the outer faces, which leaves $r^4$ states unchanged. There are 3 of these rotations.
\item[120 Degree Rotation About a Corner:] There are 2 fixed corners in this rotation, and each is associated with 3 faces, which must be the same color. This leaves $r^2$ states unchanged. There are 8 of these rotations.
\item[180 Degree Rotation About an Edge:] Every face has exactly one other face it must match with, leaving $r^3$ states unchanged. As a rotation around an edge equals a rotation about its opposite edge, there are 6 of these rotations.
\end{description}

By Burnside's lemma, the number of ways to color a cube with $r$ colors is
\[\frac{1}{24} \left( r^6 + 3r^4 + 12r^3 + 8r^2 \right)\]


\end{document}
