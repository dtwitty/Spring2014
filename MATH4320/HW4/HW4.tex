\documentclass[12pt]{article}
\usepackage{fullpage}
\usepackage{amsmath, amsthm, amssymb}
\usepackage{enumerate}
\usepackage{mathtools}
\usepackage{multicol}
\usepackage{nicefrac}
\usepackage{graphicx}
\usepackage{hyperref}

\usepackage{float}
\newtheorem{lemma}{Lemma}
\newtheorem{theorem}{Theorem}
\newtheorem{claim}{claim}
\newtheorem*{lemma*}{Lemma}
\newtheorem*{theorem*}{Theorem}
\newtheorem*{claim*}{Claim}
\usepackage{subcaption}

\newcommand*{\Real}{\mathbb{R}}
\newcommand*{\Z}{\mathbb{Z}}
\newcommand*{\ord}{\mathrm{ord}}
\newcommand*{\inv}{^{-1}}


\begin{document}

\title{MATH 4320 Homework 4}
\author{Dominick Twitty}
\date{}
\maketitle

\section{The One Ring of Power(set)}
\begin{claim*}
The powerset of $X$, $P(X)$, is a commutative ring with $A + B = (A - B) \cup (B - A)$ and $AB = A \cap B$.
\end{claim*}
\begin{proof}
First, we show that $P(X)$ is an abelian group under addition. Trivially, we see that $P(X)$ is closed under addition. We have that the additive identity element $0$, is the the empty set. Set union is commutative, therefore addition is commutative. Next, we see that $a \in (A + B) \iff a \in A \oplus B$, and as $\oplus$ is associative, addition is associative. Finally, we have that for any subset $A$, $A$ is also the unique additive inverse.

Next, we investigate multiplication. We easily see that multiplication is associative and commutative. Finally, as conjunction distributes over exclusive-or, we see that multiplication distributes over addition.

The given set and operations therefore satisfy all requirements of a commutative ring.
\end{proof}

\section{Subring of the Rationals}
\begin{claim*}
Let $D \in \mathbb{Q}$ not be a perfect square. $\mathbb{Q}(\sqrt{D}) = \{a + b\sqrt{D} | a, b \in \mathbb{Q}\}$ is a subring of $\mathbb{C}$.
\end{claim*}
\begin{proof}
To show that $\mathbb{Q}(\sqrt{D})$ is a subring, we must show that it is a subgroup under addition, is closed under multiplication, and contains the multiplicative identity. First, we see that it contains the additive inverse $0 = 0 + 0\sqrt{D}$, and the multiplicative inverse $1 = 1 + 0\sqrt{D}$. Next, we show that $\mathbb{Q}(\sqrt{D})$ is closed under addition.
\begin{align*}
(a + b\sqrt{D}) + (c + d\sqrt{D}) = (a + c) + (b + d)\sqrt{D}
\end{align*}
Next, we show that it is closed under multiplication. Remember that $D$ is rational.
\begin{align*}
(a + b\sqrt{D}) * (c + d\sqrt{D}) = (ac + bdD) + (ad + bc)\sqrt{D}
\end{align*}
Finally, we see that $\mathbb{Q}(\sqrt{D})$ contains additive inverses for all elements. For some $x = a + b\sqrt{D}$, let $-x = (-a) + (-b)\sqrt{D}$. We have now shown that $\mathbb{Q}(\sqrt{D})$ meets all requirements of a subring.
\end{proof}

\pagebreak
\begin{claim*}
$\mathbb{Q}(\sqrt{D})$ is a field.
\end{claim*}
\begin{proof}
We must show that every non-zero element of $\mathbb{Q}(\sqrt{D})$ has a multiplicative inverse. It can be shown that every element of the form $a + b\sqrt{D}$ has an inverse
\[\frac{a - b\sqrt{D}}{a^2 -b^2D}\]
We see that both coefficients are rational as $D$ is rational, and is defined as long as $a^2 \neq b^2D$. If this were the case, then we'd have  that
\[a^2 = b^2D \implies D = \frac{a^2}{b^2}\]
Then we would have a contradiction, as we stated that $D$ is not a perfect square. Therefore, this form of inverse is defined for all $a, b$. As every nonzero element has an inverse, $\mathbb{Q}(\sqrt{D})$ is a field.
\end{proof}

\section{Finite Integral Domains are Fields}
Let $a$ be a non-zero member of $R$. First, we define a mapping $f_a : R \rightarrow R$ with $f_a(x) = ax$. As $R$ is an integral domain, we know that if $f_a(x) = 0$, then $a = 0$ or $x = 0$. As we said $a$ is nonzero, $x$ must be zero. So, the kernel of $f_a$ is $\{0\}$, and as $f_a$ is a group homomorphism, $f_a$ is injective.

By the pigeonhole principle, an injective mapping on a between two finite sets of the same size is surjective, so $f_a$ is also surjective. As $f_a$ is surjective, there exists some $a\inv \in R$ with $f_a(a\inv) = 1$. Therefore, $a$ must have an inverse. Therefore, every finite integral domain is also a field. 
\section{$RG$ a Commutative Ring}
First, we show that $RG$ is an abelian group under addition. As $R$ is closed under addition, $RG$ is also closed under addition. Next, we see that $RG$ contains an additive identity element $0 = 0_R0_G$. The additive inverses are of the form
$-(r_1g_1 + \ldots + r_ng_n) = (-r_1g_1 + \ldots + -r_ng_n)$.

Next we have that the multiplicative identity of $RG$ is $1 = 1_R1_G$. As $R$ is commutative, one can see that $RG$ is also commutative. Given $ p \in RG$, let $p_i$ be coefficient of $g_i$ in $p$. We show that multiplication distributes over addition
\begin{align*}
(p(q + r))_n = \sum_{g_i g_j = g_n} p_i (q + r)_j = \sum_{g_i g_j = g_n} p_i (q_j + r_j) =  \sum_{g_i g_j = g_n} \left(p_i q_j + P_i r_j\right)
\end{align*}
Using the same technique, we show associativity on $p, q, r \in RG$
\begin{align*}
(p(qr))_n = \sum_{g_i g_j = g_n} p_i (qr)_j = \sum_{g_i g_j = g_n} p_i \sum_{g_k g_l = g_j} q_k r_l = \sum_{g_i g_j = g_n} \sum_{g_k g_l = g_j} p_i (q_k r_l) = \sum_{g_i g_j g_k = g_n}  p_i (q_j r_k)
\end{align*}
Notice in the last substitution that we fused sums over 1 and 2 variables into a single sum over 3 variables. As $R$ is associative, we have that $p_i(q_j r_k) = (p_i q_j)r_k$, which implies $(p(qr))_n = ((pq)r)_n$. By definition similar to polynomial equality (coefficients match for each matching $g_i$), we have that $p(qr) = ((pq)r)$ and $RG$ is a commutative ring.


\section{$F((x))$ is a Field}
One can easily see that $F((x))$ is closed under addition and multiplication - as addition and multiplication of finitely many finite integers will never produce an infinite result, we have that the addition or product of two members of $F((x))$ will have finitely many negative powers. Therefore, $F((x))$ is a subring of $F(x)$. $F((x))$ is commutative and associative for the same reason that $F[[x]]$ is.

What is left to show is that $F((x))$ is closed under inversion. Suppose we have some $f = f_Nx^N + f_{N+1}x^{N+1} + \ldots$ in $F((x))$. Consider $g = g_Nx^0 + f_{N+1}x^1 + \ldots$ in $F((x))$. We may write $f = x^Ng$. We also see that $g$ is in $F[[x]]$. We see that as $F$ is a field, the leading coefficient of $g$ is invertible, and by proposition 7.5 of \cite{wagner}, $g$ is invertible in $F[[x]]$. Consider some $h = x^{-N}g\inv$. Then we have 
\begin{align*}
fh = x^N g x^{-N} g\inv = 1
\end{align*}
Therefore, by the existence of inverses in $F[[x]]$, we have that inverses exist in $F((x))$, and $F((x))$ is a field.
\section{11 is not Prime in $\mathbb{Z}(\sqrt{5})$}
\begin{proof}
Let $x = a + b\sqrt{5}$ and $y = c + d\sqrt{5}$. We wish to show that 11 divides $xy$, then 11 divides neither $x$ nor $y$. First, the form of elements divisible by 11 is

\[(11 + 0\sqrt{5})(j + k\sqrt{5}) = 11j + 11k\sqrt{5} \]

So if 11 divides $a + b\sqrt{5}$, then 11 divides both $a$ and $b$. If 11 divides $xy$, then
\begin{align*}
11 | (ac + 5bd) & &\text{and}& & 11 | (ad + bc)
\end{align*}
We must show that there exist $a$, $b$, $c$, and $d$ such that the above conditions hold, but 11 does not divide $a$ or $b$ and $c$ or $d$. One example of this situation is $(a, b, c, d) = (1,3,1,8)$.
It is enough to show that 11 has no multiplicative inverse in $\mathbb{Z}(\sqrt{5})$. We have that $\nicefrac{1}{11}$ is rational, but $a + b\sqrt{5}$ is not rational for $b \neq 0$. Otherwise it could be shown that $\sqrt{5}$ is rational. So, 11 does not divide $a$ or $c$, and therefore divides neither $x$ nor $y$. Therefore, 11 is not prime in $\mathbb{Z}(\sqrt{5})$.
\end{proof}

\section{Radical of an Ideal is an Ideal}
\begin{proof}
First, we have that $I \subseteq \sqrt{I}$. We show that $\sqrt{I}$ is closed under addition. Consider $a^n, b^m \in I$. Then
\[(a + b) ^ {m + n} = \sum_{k = 0} ^{m + n} \binom{m + n}{k} a^k b^{m + n - k}\]
Because $I$ is an ideal, if $k \geq n$, then $a^k \in I$, otherwise $b ^ {m + n - k} \in I$. Every term in the expansion of the sum is in $I$, therefore the entire sum is in $I$, and $\sqrt{i}$ is closed under addition. Next we show that $\sqrt{I}$ is absorbs multiplications. For $r \in R$, $(ra)^n = r^n a^n$ is in $I$. Therefore, if $a \in \sqrt{I}$, $ra \in \sqrt{I}$. Finally, we must show that $\sqrt{I}$ is closed under additive inversion to show that it is a subgroup. Consider $a^n \in I$. If $n$ is even, then $(-a)^n = a^n \in I$. If $n$ is odd, $(-a)^{n + 1} = a^{n + 1} = a^n a \in I$. Therefore, $\sqrt{I}$ is closed under addition and is an ideal.
\end{proof}

\section{$(x, y)$ is not principle in $\mathbb{Q}[x,y]$}
\begin{proof}
First, we see that every member of $(x, y)$ must have a zero constant term. Suppose $(x, y)$ were also equal to $(p)$. Both $x^1$ and $y^1$ must then be divisible by $p$. This can only hold if $p$ is a nonzero constant (namely 1). However, zero is the only constant in $(x, y)$, and we have a contradiction. Therefore, $(x, y)$ is not principle. 
\end{proof}


\section{An ideal $I$ of $\mathbb{Z}$ is prime if and only if $I = (\text{prime } p)$}
\begin{proof}
First, we see that $(p)$ is the set of all integers divisible by $p$. By Euclid's lemma, if $ab \in (p)$,  $a$ or $b$ must be in $(p)$. We also have that if $p | a$ and $p | b$, $p | (a + b)$. Finally, if $p | a$, then $p | ab$, so $(p)$ is a prime ideal.

Consider the other direction. Let $d$ be the smallest positive element of $I$. Consider $x \in I$. We divide $x$ by $d$ to get $x = qd + r$, with $0 \leq r < d$. Equivalently, $r = x - qd$. Because $I$ is an ideal, we know that both $x$ and $qd$ are in $I$, thus $r$ is in $I$. In the division algorithm, $0 \leq r < d$, but $d$ is minimal, so $r = 0$ and $I = (d)$. Let $d = ab$. Then $d | ab$ but $d \nmid a$ and $d \nmid b$. Therefore, $d$ is prime.
\end{proof}

\section{$\sqrt{P} = P$ for Prime Ideal $P$}
\begin{proof}
Let $x \in \sqrt{P}$. For some minimal $n \geq 1$, $x ^ n \in P$. If $n \geq 2$, $x x^{n - 1} \in P$. As $P$ is prime, either $x \in P$ or $x ^ {n - 1} \in P$. As $n$ is minimal, $x ^ {n - 1} \notin P$, so $x \in P$. 
\end{proof}

\begin{thebibliography}{2}
\bibitem{wagner} \url{http://www.math.uwaterloo.ca/~dgwagner/co430I.pdf}
\end{thebibliography}


\end{document}
