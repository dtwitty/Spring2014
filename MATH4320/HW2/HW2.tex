\documentclass[12pt]{article}
\usepackage{fullpage}
\usepackage{amsmath, amsthm, amssymb}
\usepackage{enumerate}
\usepackage{mathtools}
\usepackage{multicol}
\usepackage{nicefrac}

\newtheorem{lemma}{Lemma}
\newtheorem{theorem}{Theorem}

\newcommand*{\Real}{\mathbb{R}}

\begin{document}
\title{MATH 4320 Homework 2}
\author{Dominick Twitty}
\maketitle

\begin{enumerate}
\item
\begin{enumerate}[(a)]
\item $\Real^{\geq 0}$ with operation $a \star b = \max(a,b)$ is not a group. We have that $\max(a, 0) = \max(0, a) = a$, which means the identity element $e = 0$. However, for any $a > 0$, there is no $b$ such that $\max(a, b) = 0$. There is no inverse element, so this is not a group.

\item $\Real^\times$ with operation $a \star b = \frac{a}{b}$ is not a group. We attempt to verify associativity:
\begin{align*}
a \star (b \star c) &= (a \star b) \star c\\
\frac{a}{\nicefrac{b}{c}} &= \frac{\nicefrac{a}{b}}{c}\\
\frac{ac}{b} &\neq \frac{a}{bc}
\end{align*}

\item $\Real$ with operation $a \star b = \text{frac}(a, b) = ab - \lfloor ab \rfloor$ is not a group. We have an operation that is a function $\star : \Real \times \Real \rightarrow [0, 1)$. If $a$ is greater than 1, there is no $b$ such that $\text{frac}(a, b) = a$, as $a$ is not in the codomain of $\star$. Therefore, the group is not closed. 

\item The cartesian product $A \times B$ where $A$ and $B$ are groups, which we call $G$, with operation $(a, b) \star (c, d) = (ab, cd)$ is a group. Because $A$ and $B$ are closed, we have that operation $\star : (A \times B) \times (A \times B) \rightarrow A \times B$, we know that $G$ is closed.

We show associativity:
\begin{align*}
(a,b) \star ((c, d) \star (e, f)) &= ((a,b) \star (c, d)) \star (e, f)\\
(a, b) \star (ce, df) &= (ac, bd) \star (e, f)\\
(ace, bdf) &= (ace, bdf)
\end{align*}

Let $e$ be the identity element of $A$ and $f$ be the identity element of $B$. We have
\begin{align*}
(e, f) \star (a, b) &= (a,b) \star (e, f)\\
(ea, fb) &= (ae, bf)\\
(a, b) &= (a, b)
\end{align*}

So $(e, f)$ is the identity element of $G$. Finally, we show inverse elements:
\begin{align*}
(a, b) \star (a^{-1}, b^{-1}) &= (a^{-1}, b^{-1}) \star (a, b)\\
(aa^{-1}, bb^{-1}) &= (a^{-1}a, b^{-1}b)\\
(e, f) &= (e, f)
\end{align*}

$G$ meets all the requirements of a group.

\end{enumerate}

\item We have that $G$ is a finite group, and we wish to show that $\text{ord}(a) < \infty$ for all $a$ in $G$. That is, we wish to show that there exists some finite $m > 0$ such that $a^m = e$. 

\begin{lemma}
For $a$ in finite group $G$, there exists $m \neq n$ such that $a^n = a^m$.
\end{lemma}
\begin{proof}
By the closure property of groups, we have that $a^k \in G$. By the pigeonhole principle, the list $a^1, a^2, a^3,\ldots$ must eventually repeat. If the list did not repeat, $G$ could not be closed unless $G$ were infinite. Because the list repeats, there must exist two powers that equal each other.
\end{proof}

\begin{lemma}
For $a$ in finite group $G$, $a$ must have some idempotent power. That is, there exists some $k$ such that $a^k a^k = a^k$.
\end{lemma}
\begin{proof}
By the previous lemma, we have some $m \neq n$ with $a^m = a^n$. We set $m = k$ and $n = k + l$. We have:


\end{proof}

\begin{lemma}
Every $a$ in finite group $G$ has an idempotent power. That is, there exists some $k$ such that $a^k a^k = a^k$.
\end{lemma}

\item 
\begin{lemma}
If $G$ is a group and for all $x$ in $G$ we have $x^2 = 1$, then $G$ is abelian.
\end{lemma}
\begin{proof}
Showing that a group is abelian is equivalent to showing $xy = yx$.
\begin{align*}
xy &= yx\\
y(xy)x &= y(yx)x\\
(yx)(yx) &= (yy)(xx)\\
(yx)^2 &= y^2 x^2\\
1 &= 1
\end{align*}

\end{proof}


\end{enumerate}


\end{document}
